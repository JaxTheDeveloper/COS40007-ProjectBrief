\section{Requirements}
\subsection{Must-Have Features}
\begin{frame}
  \frametitle{3.1 Must-Have Features (Core Models)}

        \textbf{Activity Classification Model:} Classify (boning, slicing, cutting, idle) from 1-min sensor data.

        \textbf{Knife Sharpness Model:} Classify condition (Sharp, Medium, Blunt).

        \textbf{Data Pipeline:} Robust feature extraction from raw sensor data.

        \textbf{Ensemble System:} Combined output for both activity and sharpness.

    

\end{frame}

\begin{frame}{Ensemble system}
    Random forest will be chosen first due to its high accuracy seen on Portfolio 3.
    \begin{block}{Ensemble learning}
        \includegraphics[width=\textwidth]{ensemble learning.png}
    \end{block}
\end{frame}

\subsection{Optional Features}
\begin{frame}
  \frametitle{3.2 Optional Features (Stretch Goals)}
  \begin{itemize}
    \item \textbf{Advanced UI Dashboard:}
    \begin{itemize}
        \item Industrial-grade GUI (Python `tk` module).
        \item Role-based access (Worker vs. Supervisor).
        \item Worker "clock-in" and activity logging.
        \item Supervisor reports on worker performance.
    \end{itemize}
    \item \textbf{Real-Time Capability:} Detect activity/sharpness from live sensor streams.
    \item \textbf{Visualization Tools:}
    \begin{itemize}
        \item Graphs of sensor data (e.g., acceleration over time).
        \item Real-time "stick man" visualization of worker.
    \end{itemize}
  \end{itemize}
\end{frame}

\subsection{Deployment Plan}
\begin{frame}
  \frametitle{3.3 Deployment Plan}

    \begin{block}{Weeks 7-9: CSV-fed input (real-time sensor simulation)}
    \begin{itemize}
        \item Follows producer-consumer architectural pattern.
        \item \textbf{Producer:} Read and serialize the data from dataset's \emph{eval} set.
        \item \textbf{Consumer:} Receives the serialized data and fed into the ensemble model.
        \item Predictions are expected to be in real-time.
    \end{itemize}
    \end{block}

    \begin{block}{Weeks 10-12: Migrates to sensor-read data}
    \begin{itemize}
        \item Overall architecture is kept \textit{unscathed}.
        \item I2C-based accelerometers will be connected and read by a Raspberry Pi (due to its support of multiple I2C ports on the board).
        \item Acceleration (xyz) values shall be transferred in terms of PWM (or angle). Accelerometer values may be in 12-bit (4096) resolution.
    \end{itemize}
    \end{block}
    

\end{frame}